\documentclass[12pt]{article}
\usepackage[utf8]{inputenc}
\usepackage[brazil]{babel}
\usepackage[margin=.9in]{geometry}
\usepackage{url}
\usepackage{amsmath}
\usepackage{longtable}

\title{Escolha do tema para o projeto de Tradutores: Linguagem CStr}
\author{Eduardo Furtado --- 09/0111575\\Departamento de Ciência da Computação\\Universidade de Brasília}

\begin{document}

\maketitle
	
\section{Área da Ciência da Computação}

\indent

A área escolhida é algoritmos e estruturas de dados em strings. Propõe-se a implementação de um tradutor da linguagem C (CStr) para código de 3 endereços que implemente o tipo primitivo string com operadores e métodos auxiliares. \\

\section{Descrição da linguagem formal para a qual será implementado o tradutor}

\indent

Para fazer o processo de tradução utilizarei ferramentas como o bison e o flex, bem como o gcc para compilar o tradutor que será desenvolvido em linguagem C.

Este trabalho apresenta uma linguagem CStr, a gramática de CStr é baseada em uma versão reduzida da linguagem C \cite{minic}, com acréscimo do tipo primitivo string e o operador de concatenação (.), e métodos segundo \cite{cplusplus}.

A gramática livre de contexto a seguir descreve a linguagem CStr proposta, onde variáveis (simbolos não-terminais) começam com letras maiúsculas, Function é a variável inicial e todos os outros simbolos são terminais. A barra vertical $|$ é usada para indicar definições alternativas para um não-terminal.\\

\begin{longtable}{ r l }
%%%%	chamada a subrotinas (funções, procedimentos, subrotinas):
	Function 		$\rightarrow$ 	& Type Identifier ( FormalArgList ) CompoundStmt \\
	Identifier		$\rightarrow$   & char ( char $|$ digit $|$ \_ )* \\
	FormalArgList 	$\rightarrow$ 	& FormalArg \\
									& $|$ FormalArgList , FormalArg \\
	FormalArg 		$\rightarrow$ 	& Type Identifier \\
	ArgList			$\rightarrow$ 	& Arg \\
									& $|$ ArgList , Arg \\
	Arg				$\rightarrow$ 	& Identifier \\
	Declaration		$\rightarrow$ 	& Type IdentList ; \\
	Type			$\rightarrow$ 	& int \\
									& $|$ float \\
									& $|$ string \\
	%StringReference $\rightarrow$	& Identifier $|$ StringConstant \\	
	StringFunction	$\rightarrow$	& string . Identifier () \\
									& $|$ string . Identifier ( ArgList ) \\
	StringConcat	$\rightarrow$	& string . string \\
	IdentList		$\rightarrow$ 	& Identifier , IdentList \\
									& $|$ Identifier \\
	Stmt 			$\rightarrow$ 	& WhileStmt \\
%								  	& $|$ ForStmt \\
									& $|$ Expr ; \\
									& $|$ IfStmt \\
									& $|$ CompoundStmt \\
									& $|$ Declaration \\
									& $|$ StringFunction \\
									& $|$ StringConcat \\
									& $|$ ; \\
%	ForStmt			$\rightarrow$ 	& for ( Expr ; OptExpr ; OptExpr ) Stmt \\
%	OptExpr			$\rightarrow$ 	& Expr \\
%									& $|$ $\varepsilon$ \\
%%%%	comando de repetição (while, until ou for):
	WhileStmt		$\rightarrow$ 	& while ( Expr ) Stmt \\
%%%%	comando condicional (if, com tratamento de expressões booleanas):
	IfStmt			$\rightarrow$ 	& if ( Expr ) Stmt ElsePart \\
	ElsePart		$\rightarrow$ 	& else Stmt \\
									& $|$ $\varepsilon$ \\
	CompoundStmt	$\rightarrow$ 	& \{ StmtList \} \\
	StmtList		$\rightarrow$ 	& StmtList Stmt \\
									& $|$ $\varepsilon$ \\
	Expr			$\rightarrow$ 	& Identifier = Expr \\
									& $|$ Rvalue \\
	Rvalue			$\rightarrow$ 	& Rvalue Compare Mag \\
									& $|$ Mag \\
	Compare			$\rightarrow$ 	& ==  \\
									& $|$ $<$ \\
									& $|$ $>$ \\
									& $|$ $<=$ \\
									& $|$ $>=$ \\
									& $|$ $!=$ \\
%%%%	tratamento de expressões aritméticas (com pelo menos dois tipos distintos de dados):										
	Mag				$\rightarrow$ 	& Mag + Term \\
									& $|$ Mag - Term \\
									& $|$ Term \\
	Term			$\rightarrow$ 	& Term * Factor \\
									& $|$ Term / Factor \\
									& $|$ Factor \\
	Factor			$\rightarrow$ 	& ( Expr ) \\
									& $|$  - Factor \\
									& $|$  + Factor \\
									& $|$  Identifier \\
									& $|$  number \\ \\
\end{longtable}
		

A ordem em que as operações aparecem determina a precedência de cada operador, onde a primeira tem menor procedência e a última tem a maior procedência.


\section{Motivação para a escolha da linguagem}

\indent

Quando se compara C com outras linguagens de programação, como por exemplo C++, fica evidente que uma das facilidades que o C não tem é no tratamento de strings, algo extremamente abrangente na computação.

Entretanto, C é uma linguagem extremamente usada, e sempre que um programador precisa manipular strings, passa por sua cabeça que seria mais fácil estar utilizando outra linguagem.

Este trabalho pode ser útil ao programador da linguagem e ao usuário leigo para facilitar e agilizar a programação em C quando é necessário manipular strings, disponibilizando strings como um novo tipo primitivo e fornecendo operadores e métodos para esse tipo.

\section{Descrição breve da semântica da linguagem}

\indent

O tradutor será implementado de acordo com o que for aprendido nas aulas e na bibliografia da matéria, bem como bibliografia oficial do manual do C. \cite{compiladores2008}, \cite{gnuc}, \cite{yacc1995}.

Espera-se implementar:
\begin{itemize}
	\item String como tipo nativo;
	\item Operador . para concatenação de strings;
	\item Funções apresentadas em \cite{cplusplus};
	\item String matching de maneira eficiente segundo \cite{clrs09};
\end{itemize}

%\begin{itemize}
%	\item String como tipo nativo;
%	\item Operador . para concatenação de strings;
%	\item Método strcount para contar a ocorrência de uma string em outra string;
%	\item Método strfind para encontrar a primeira ocorrência de uma string em outra string;
%	\item Método strreplace para substituir todas as ocorrências de uma cadeia em uma string;
%	\item Método strexplode para divisão de uma string.
%\end{itemize}

Ou seja, visa-se encorporar no C uma parte de orientação a objetos para o tipo String. 


Abaixo está um exemplo de declaração de uma variável string:

\begin{verbatim}
	string nome_da_variavel_string = "Isso é uma variável em C do tipo nativo string!";
\end{verbatim}

A partir disso, será possível concatenar strings:
\begin{verbatim}
	string s = "Isso é uma variável em C do tipo nativo string!";
	string aux = "!";
	nome_da_variavel_string . "!";
	nome_da_variavel_string . aux;
\end{verbatim}

\begin{verbatim}
	string nome_da_variavel_string = "Isso é uma variável em C do tipo nativo string!";
	string aux = "!";
	nome_da_variavel_string . "!";
	nome_da_variavel_string . aux;
\end{verbatim}

Além disso, utilizar métodos, como o de matching proposto em \cite{clrs09}:
\begin{verbatim}
	string s = "Achar uma agulha num palheiro";
	string palheiro = s.kmpPreprocess();
	string result = palheiro.kmpSearch("agulha");
\end{verbatim}

%XNão será necessário declarar o tamanho da variável. Inicialmente será alocado espaço suficiente para a string, 


%%As estruturas de linguagens de programação requeridos (condicionais, laços, expressões lógicas e procedimentos) já fazem parte da linguagem C.

\bibliographystyle{unsrt}
\bibliography{main}

\end{document}